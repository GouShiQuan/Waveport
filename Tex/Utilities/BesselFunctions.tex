
\section{Spherical Bessel Functions}

\label{sphericalbess}
Matlab does not have built-in routines for spherical Bessel functions. Below are several variants of spherical Bessel functions and derivatives encountered in scattering problems.

%Log derivatives. 

%\bgroup
%\def\arraystretch{3}
%\begin{table}[htdp]
%\caption{default}
%\begin{center}
%\begin{tabular}{|c|c|c|}
%\hline
%Function & Equation & Comments \\
%\hline
%\texttt{jinc(x)}& $\textrm{jinc}(x) = \dfrac{J_1(x)}{x}$ & $\lim_{x\rightarrow0} \textrm{jinc}(x) = \dfrac{1}{2}$ \\
%\hline
%\texttt{sbesselj(n,x)} & $j_n(x) = \sqrt{\dfrac{\pi}{2x}}J_{n+1/2}(x)$ & $\lim_{x\rightarrow0}j_n(x) = \left\{ \begin{array}{c} 1, \quad n=0 \\ 0, \quad n \ge 1\end{array}  \right.$  \\
%\hline
%\end{tabular}
%\end{center}
%\label{default}
%\end{table}
%\egroup

\addtocontents{toc}{\protect\setcounter{tocdepth}{1}}
\subsection{Sombrero function: $\textrm{jinc}(x)$}
\addtocontents{toc}{\protect\setcounter{tocdepth}{2}}

The \textrm{jinc}(x) function, or Sombrero function, is defined:
\begin{equation}
\textrm{jinc}(x) = \dfrac{J_1(x)}{x}
\end{equation}
%
%\noindent where
%\begin{equation}
%\lim_{z\rightarrow0} \textrm{jinc}(z) = \dfrac{1}{2}
%\end{equation}

\noindent where small arguments are computed with the Taylor series 
\eq{ \textrm{jinc}(x) \approx \dfrac{1}{2} - \dfrac{x^2}{16} + O(x^4)}

{\footnotesize
\VerbatimInput{\code/BesselFunctions/jinc.m}
}

%This is slow in Matlab and we'd like to avoid the division.  Start with the series representation for the Bessel function of complex argument
%\eq{J_{\nu}(z) = \left(\dfrac{z}{2}\right)^\nu \sum_{k=0}^{\infty} \dfrac{ (-1)^k \left( \dfrac{z}{2}\right)^{2k} }{ k! \Gamma(\nu + k + 1)}}
%
%Evaluating $\nu =1$, and dividing by $z$, 
%\eq{\textrm{jinc}(z) = \dfrac{1}{2} \sum_{k=0}^{\infty}(-1)^k a_k }
%\eq{a_k = \dfrac{  \left( \dfrac{z}{2}\right)^{2k} }{ k! \Gamma(k + 2)} }
%
%The log transform is used to avoid multiplying and dividing by large and small numbers, facilitate fast recursive of the factorials, and take care of the limit when $z = 0$.  
%\eq{a_k = e^{\ln a_k}}
%
%where, using $\Gamma(n) = (n-1)!$, 
%\ea{\ln a_k &=& 2k (\ln z - \ln 2) - \ln k! - \ln (k + 1)! }
%
%Evaluating at $k+1$, separating a factor of $\ln a_k$, and shifting the index down by 1, we get the recursion for the $k$th term of the series
%\ea{\ln a_{0} &=& 0 \\
%\ln a_{k} &=& \ln a_{k-1} + 2 (\ln z - \ln 2) - \ln k - \ln (k + 1) }
%
%%\ea{\ln a_{k+1} &=& 2(k+1) (\ln z - \ln 2) - \ln (k+1)! - \ln (k + 2)! \\
%%\ &=& 2k (\ln z - \ln 2) + 2 (\ln z - \ln 2) - (\ln k! + \ln (k+1)) - ( \ln (k + 1)!  + \ln (k + 2) ) \\
%%\ &=& \ln a_k + 2 (\ln z - \ln 2) - \ln (k+1) - \ln (k + 2) }
%
%The code is as follows.  

\addtocontents{toc}{\protect\setcounter{tocdepth}{1}}
\subsection{Spherical Bessel function: $j_n(x)$}
\addtocontents{toc}{\protect\setcounter{tocdepth}{2}}

The spherical Bessel function is 

\begin{equation}
j_n(x) = \sqrt{\dfrac{\pi}{2x}}J_{n+1/2}(x)
\end{equation}

\noindent where

\begin{equation}
\lim_{x\rightarrow0}j_n(x) = \left\{ \begin{array}{c} 1 - \dfrac{x^2}{6} + O(x^4) \quad n=0 \\ 0, \quad n \ge 1\end{array}  \right.
\end{equation}

The routine \texttt{sbesselj} returns the spherical Bessel function for matching arrays of $n$ and $x$ in the format of \texttt{besselj}. It uses the Taylor series near $x = 0$ for $n=0$, and substitutes 0 when $x=0$ for $n>1$.  

{\footnotesize
\VerbatimInput{\code/BesselFunctions/sbesselj.m}
}

\addtocontents{toc}{\protect\setcounter{tocdepth}{1}}
\subsection{Spherical Bessel function: $j_n(x)/x$}
\addtocontents{toc}{\protect\setcounter{tocdepth}{2}}


This variant of the spherical Bessel function has $j_n(x)$ divided by $x$: 
\begin{equation}
\dfrac{j_n(x)}{x} = \dfrac{1}{x^{3/2}}\sqrt{\dfrac{\pi}{2}}J_{n+1/2}(x)
\end{equation}

\noindent where
\begin{equation}
\lim_{x\rightarrow0}j_n(x) = \left\{ \begin{array}{c} \infty, \quad n=0 \\ \dfrac{1}{3} - \dfrac{x^2}{30} + O(x^4), \quad n = 1 \\ 0, \quad n \ge 2\end{array}  \right.
\end{equation}

This is computed in the routine \texttt{sbesselj2}, which takes the same inputs as \texttt{sbesselj}.  We let the routine return \texttt{Inf} for $x=0$ when $n=0$.  This variant is found in electromagnetic scattering when usually the monopole term ($n=0$) is not needed.

{\footnotesize
\VerbatimInput{\code/BesselFunctions/sbesselj2.m}
}

\addtocontents{toc}{\protect\setcounter{tocdepth}{1}}
\subsection{Spherical Bessel function derivative: $j_n'(x)$}
\addtocontents{toc}{\protect\setcounter{tocdepth}{2}}

The derivative of the spherical Bessel function with respect to $x$ is given by the recurrence relation
\begin{equation}
j_n'(x) = -j_{n+1}(x) + \dfrac{n}{x}j_n(x)
\end{equation}

This is computed in \texttt{sbesseljp} with \texttt{sbesselj} and \texttt{sbesselj2}, which takes care of $x=0$. 

{\footnotesize
\VerbatimInput{\code/BesselFunctions/sbesseljp.m}
}

\clearpage
\addtocontents{toc}{\protect\setcounter{tocdepth}{1}}
\subsection{Spherical Bessel function derivative: $[xj_n(x)]'$}
\addtocontents{toc}{\protect\setcounter{tocdepth}{2}}

This variant occurs often enough in scattering to have its own function, \texttt{sbesseljp2}, 
\ea{[xj_n(x)]' &=& j_n(x) + x j'_n(x) \\
\ &=&  (1+n)j_n(x) - x j_{n+1}(x)}

This variant is found in the Mie scattering solution for spheres.

{\footnotesize
\VerbatimInput{\code/BesselFunctions/sbesseljp2.m}
}


\addtocontents{toc}{\protect\setcounter{tocdepth}{1}}
\subsection{Spherical Hankel function: $h_n^{(1)}(x)$}
\addtocontents{toc}{\protect\setcounter{tocdepth}{2}}

The spherical Hankel function is given by 
\begin{equation}
h_n^{(1)}(x) = \sqrt{\dfrac{\pi}{2x}}H_{n+1/2}(x)
\end{equation}

This is always irregular at the origin, and computed in the routine \texttt{sbesselh}

{\footnotesize
\VerbatimInput{\code/BesselFunctions/sbesselh.m}
}

\addtocontents{toc}{\protect\setcounter{tocdepth}{1}}
\subsection{Spherical Hankel function derivative: ${h'}_n^{(1)}(x)$}
\addtocontents{toc}{\protect\setcounter{tocdepth}{2}}

The spherical Hankel derivative is computed in the routine \texttt{sbesselhp}
\begin{equation}
{h'}_n^{(1)}(x) = -h_{n+1}^{(1)}(x) + \dfrac{n}{x}h_n^{(1)}(x)
\end{equation}


{\footnotesize
\VerbatimInput{\code/BesselFunctions/sbesselhp.m}
}

\addtocontents{toc}{\protect\setcounter{tocdepth}{1}}
\subsection{Spherical Hankel function derivative: $[xh_n^{(1)}(x)]'$}
\addtocontents{toc}{\protect\setcounter{tocdepth}{2}}

Similar to before, this variant is computed in the routine \texttt{sbesselhp2}
\ea{[xh_n^{(1)}(x)]' &=& h_n^{(1)}(x) + x {h'}_n^{(1)}(x) \\
\ &=&  (1+n)h_n^{(1)}(x) - x h_{n+1}^{(1)}(x)}

This variant is found in the Mie scattering solution for spheres.


{\footnotesize
\VerbatimInput{\code/BesselFunctions/sbesselhp2.m}
}

